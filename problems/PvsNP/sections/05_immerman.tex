\section{Immerman--Szelepcsényi completeness}\label{P:immerman}

As an auxiliary result we verify the closure of \(\mathbf{NL}\) under
complement following the method of Immerman and Szelepcsényi.  Although not
strictly necessary for the main separation, the argument illustrates the
nondeterministic counting technique that underlies our verifier.

\begin{lemma}[Reachability counting]\label{lem:reach-count}
Let \(G\) be a directed graph on \(n\) vertices and \(s,t\in V(G)\).  The
number of vertices reachable from \(s\) can be computed by a nondeterministic
logspace machine using at most \(O(\log n)\) bits of auxiliary storage.
\end{lemma}

\begin{proof}[Idea]
The machine recursively guesses frontier sets and verifies their sizes by
iterating over the adjacency list.  Details follow the classical proof; we
omit them here for brevity.
\end{proof}

\begin{theorem}[Immerman--Szelepcsényi]
\(\mathbf{NL}=\mathbf{coNL}.\)
\end{theorem}

\begin{proof}[Sketch]
Apply Lemma~\ref{lem:reach-count} to the complement of the reachability
problem.  \qedhere
\end{proof} 