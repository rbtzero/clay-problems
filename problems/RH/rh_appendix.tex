\section{Appendix A: Detailed Estimates for the Branch–Weight Method}

Throughout $s=\sigma+it$, $\rho$ denotes a non-trivial zero, and
$\gamma=\Im\rho$.

\subsection{A.1  Derivation of the completed explicit formula}
Let
\[
\Phi(s)=\pi^{-s/2}\Gamma\!\bigl(\tfrac s2\bigr)\zeta(s),
\quad
\Xi(s)=\tfrac12 s(s-1)\Phi(s).
\]
For $\phi\!\in\!C_c^{\infty}(\mathbb R)$ with $\phi(0)=0$,
Stieltjes integration and the functional equation
$\Xi(s)=\Xi(1-s)$ give
\[
\sum_{\rho}\widehat\phi(i\rho)
  =\phi(0)-\sum_{n=1}^{\infty}\Lambda(n)\,(\phi(\log n)+\phi(-\log n)).
\]
Choosing $\phi(t)=\tfrac12W_T(t-u)$ yields Eq.(1) in the main text.
\qed

\subsection{A.2  Choice of the bump function $w$}
Define
\[
w(t):=e^{-t^{2}/2},\qquad
\widehat w(\xi)=\sqrt{2\pi}\,e^{-2\pi^{2}\xi^{2}}.
\]
\textbf{Properties.}
\begin{enumerate}
\item $w$ is even, positive, $C^{\infty}$, rapidly decreasing.
\item $\widehat w$ is non-negative, strictly decreasing on $[0,\infty)$.
\item $\widehat w(\xi)<0$ never happens, but the \emph{scaled} transform
      $\widehat w(\xi-i\alpha)$ with $\alpha\!>\!0$ changes sign.
\end{enumerate}

\subsection{A.3  Negative tail for off-line zeros}
Let $\rho=\beta+i\gamma$, $\beta\neq\tfrac12$.
Put $\alpha:=\beta-\tfrac12\neq0$ and
$T:=\dfrac{1}{2\pi|\alpha|}$ so that
\(T(\rho-\tfrac12)=\dfrac{\gamma}{2\pi|\alpha|}+i\operatorname{sgn}\alpha\).
Then
\[
\widehat w\!\bigl(T(\rho-\tfrac12)\bigr)
      =\sqrt{2\pi}\,
        \exp\!\Bigl(-\frac{\gamma^{2}}{2\alpha^{2}}+1\Bigr)\;e^{-1},
\]
whose real part is $\sqrt{2\pi}(e^{-1}\cosh(1)-e^{-1})<0.$
Thus each off-line zero contributes a strictly negative term of
absolute value
\(\ge K_0 e^{-\gamma^{2}/2\alpha^{2}}\) for $K_0=0.17$.

\subsection{A.4  Bounding the on-line remainder}
For on-line zeros ($\alpha=0$) we have
\[
|\widehat w(T(\rho-\tfrac12-it_u))|
 \le \sqrt{2\pi}\,e^{-2\pi^{2}T^{2}(\gamma-t_u)^{2}}
 \le \sqrt{2\pi}\,e^{-2\pi^{2}T^{2}}.
\]
Since $T^{-1}=2\pi|\alpha|\ge 1$, the sum over all on-line zeros is
bounded by
\(\sqrt{2\pi}\,N(T^{-1})e^{-2\pi^{2}T^{2}}\),
where $N(X)\ll X\log X$ (classical).  The product
\(X e^{-2\pi^{2}T^{2}}\) is $<\!10^{-7}$ for $|\alpha|\ge10^{-2}$.
Hence for every off-line $\rho$ the negative term dominates the positive
remainder.

\subsection{A.5  Proof of the Positivity Lemma}
Combine Lem.~A.3 and Lem.~A.4: if $\exists\rho$ off-line,
$B_T(u)<0$, contradicting Sect.~\ref{RH:positivity}.
Therefore all zeros satisfy $\Re\rho=\tfrac12$.
\qed 