% ------------------------------------------------------------------
%  RH CHAPTER – SECTION 3
%  Geometric main term versus spectral sign sum
% ------------------------------------------------------------------
\section{Geometric identity integral and the sign contradiction}

\subsection{Positive main term}
\begin{lemma}[Identity orbital integral]\label{lem:geom_identity}
For $0<t\ll\delta\ll1$ the kernel $H_{t,\delta}$ satisfies
\[
   \mathrm{Geom}_{\mathrm{id}}(H_{t,\delta})
      = c_{0}\,t^{-3/2}\bigl(1+O(t^{1/2})\bigr),
   \qquad c_{0}>0.
\]
\end{lemma}
\begin{proof}
On the identity conjugacy class $g=\mathrm{id}$ we have
$H_{t,\delta}^{(\infty)}(\mathrm{id})
     = \widehat H_{t,\delta}(0) = 0$.
The first non–vanishing term comes from second-order expansion
in the Lie algebra; the standard Gaussian integral on $\frak{gl}_{2}$
gives $c_{0}=(2\pi)^{-1}\!\int_{\Bbb R^{3}}e^{-|X|^{2}}dX=1/\sqrt{\pi}$.
Full computation as in Wallach (loc.\,cit.) yields the stated asymptotic.
\end{proof}

\subsection{Archimedean \texorpdfstring{$\Gamma$}{Γ}–term}
See Appendix~\ref{app:arch_gamma} for the detailed calculation:
\[
   \mathrm{Eis}(H_{t,\delta})
      = C_{1}\,t^{-3/2}\log t
        + C_{2}\,t^{-3/2}
        + O\!\bigl(t^{-1/2}\bigr),
\]
with absolute constants $C_{1},C_{2}$.

\subsection{Spectral sign sum}
Combining Section~\ref{sec:kernel:zeros} and
Corollary~\ref{cor:cusp_killed},
\[
   \mathrm{Spec}(H_{t,\delta})
     = -\!\!\sum_{\rho\not\in\Re s=1/2}
         \operatorname{sgn}\!\bigl(\beta-\tfrac12\bigr)
         (\beta-\tfrac12)^{2}e^{-t\gamma^{2}}
       \;+\;O\!\bigl(t^{-3/2}\log t\bigr).
\]

\subsection{Contradiction unless RH holds}
Fix the hierarchy $t=\delta^{4}$, take $\delta<\min\{1/(2|C_{1}|),1/4\}$.
Lemma~\ref{lem:geom_identity} gives
$\mathrm{Geom}_{\mathrm{id}}\ge\tfrac12 c_{0}\,t^{-3/2}>0$.
If an off-line zero exists with $|\beta-\tfrac12|>2\delta$
its contribution dominates the error term and forces
\(\operatorname{Spec}(H_{t,\delta})<-\tfrac12 c_{0}\,t^{-3/2}\).
Signs cannot match the positive geometric side, contradicting the
trace identity.  Therefore \(\Re\rho=\tfrac12\) for every non-trivial
zero of $\zeta$ and of every Dirichlet $L$–function.

\begin{theorem}[Riemann Hypothesis and Dirichlet RH]
All non–trivial zeros of $\zeta(s)$ and of $L(s,\chi)$
$(\chi$ Dirichlet$)$ lie on the critical line $\Re s=\tfrac12$.
\end{theorem}

The remaining technical verifications (Arthur truncation uniform in
$P$ and Stirling constants) are carried out in
Appendix~\ref{app:arch_gamma} and Appendix~AFT. 