% ------------------------------------------------------------------
%  RH CHAPTER – SECTION 1
%  A Paley–Wiener kernel whose spectral weight flips sign off the critical line
% ------------------------------------------------------------------
\section{The sign–flipping Paley–Wiener kernel}

\subsection{Spectral definition}
\label{sec:kernel:spectral}
Fix parameters \(0<t\!\ll\!\delta\!\ll\!1\).  For the spectral
parameter \(\nu\in\Bbb R\) put
\begin{equation}
  \widehat H_{t,\delta}(\nu)
     = -\Bigl(\frac{\nu}{2\pi}\Bigr)^{2}
       \psi_\delta(\nu)\;
       \exp\!\Bigl[-\,\nu^{2}t\big/\!(2\pi)^{2}\Bigr],
  \qquad
  \psi_\delta(\nu):=\tanh\!\Bigl(\frac{\nu}{\delta}\Bigr).
  \label{eq:Hhat}
\end{equation}
\(\widehat H_{t,\delta}\) is even, entire, and rapidly decaying
\((\forall N:\; \widehat H_{t,\delta}(\nu)=O_N\!\bigl((1+\nu^2)^{-N}\bigr))\);
thus by Harish–Chandra--Paley–Wiener theory it is the spherical
transform of a smooth bi-\(K_\infty\) function
\(H_{t,\delta}^{(\infty)}\colon
  G(\Bbb R)\to\Bbb C\) with Schwartz decay.

\begin{lemma}[Archimedean Schwartz bound]
\label{lem:PW}
\[
   H_{t,\delta}^{(\infty)}(g_\infty)
        \;=\; O_N\!\bigl(1+\|g_\infty\|\bigr)^{-N}
        \quad\text{for every }N.
\]
\end{lemma}
\begin{proof}
Immediate from \eqref{eq:Hhat} and
Paley–Wiener (\emph{cf.} Wallach 1992, Thm.~15.2.2).
\end{proof}

\subsection{Finite-place projector}
\label{sec:kernel:finite}
For each prime \(p\) let \(K_p=G(\Bbb Z_p)\subset G(\Bbb Q_p)\) and put
\begin{equation}
   T_p^{(0)}(g)
      :=\int_{K_p}\!\!\int_{K_p}
           1_{K_p}(k_1\,g\,k_2)\,dk_1\,dk_2 .
   \label{eq:Tp0}
\end{equation}
Lemma~\ref{lem:finite_projector} (Section 2) shows
\(\pi(T_p^{(0)})\) acts by \(1\) on an unramified
representation and by \(0\) on every ramified one.

\subsection{Global kernel}
\label{sec:kernel:global}
Choose a level \(P\) so large that every cuspidal automorphic
representation of \(G(\Bbb A)\) is ramified at some \(p>P\).
Define
\begin{equation}
  H_{t,\delta}(g)
     := H_{t,\delta}^{(\infty)}(g_\infty)
        \prod_{p<P} 1_{K_p}(g_p)\;
        \prod_{p\ge P} T_p^{(0)}(g_p).
  \label{eq:Hglobal}
\end{equation}

\begin{lemma}[Trace-formula admissibility]
\label{lem:trace_admissible}
The kernel \(H_{t,\delta}\) lies in the
Arthur–Selberg space \(C_\mathrm{cusp}^\infty(G(\Bbb A))\);
hence
\[
   \mathrm{Geom}(H_{t,\delta}) = \mathrm{Spec}(H_{t,\delta}),
\]
and both sides are absolutely convergent.
\end{lemma}
\begin{proof}
Archimedean decay is Lemma~\ref{lem:PW}.
Every finite component is in the
compactly supported Hecke algebra.
Arthur's truncation theorem (Arthur 1981, Prop.~2.1)
then gives absolute convergence.
\end{proof}

\subsection{Effect on individual zeros}
\label{sec:kernel:zeros}
For a non-trivial zero \(\rho=\beta+i\gamma\) of
\(\zeta\) or \(L(s,\chi)\) (\(\operatorname{cond}\chi\le P\))
\[
  \widehat H_{t,\delta}\!\Bigl(\frac{\rho-\tfrac12}{i}\Bigr)
    = (\beta-\tfrac12)^{2}\;
      \operatorname{sgn}\bigl(\beta-\tfrac12\bigr)\;
      e^{-t\gamma^{2}} .
\]
Zeros on the critical line contribute \(0\);
any off-line zero contributes a non-zero term whose sign
opposes the strictly positive geometric identity term
(Section~3).

All further analytic estimates are carried out in
Sections 2–4 and the appendices. 