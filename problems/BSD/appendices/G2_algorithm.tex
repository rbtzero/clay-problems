% Appendix G.2 — Algorithmic desingularisation details

Pseudo–code and complexity analysis for the resolution algorithm will be inserted here. 

\section{G.2  Algorithm for a smooth, irreducible model}

\begin{algorithm}[H]
\caption{SmoothModel$(\xi,n)$}
\small
\begin{algorithmic}[1]
\REQUIRE Selmer class $\xi\in H^{1}(\Q,E[n])$
\STATE Construct torsor matrix $M$ from the cocycle values of $\xi$.
\STATE Form $F:=\det(uMv^{t})$ in variables $u_{ij}$.
\STATE \textbf{while} $\gcd(F,\partial F)\neq1$ \textbf{do}
    \STATE \quad Replace $F\gets F/\gcd(F,\partial F)$.
\STATE \textbf{for each} rational singular point $P\in V(F)$ \textbf{do}
    \STATE \quad Blow up $P$ (affine chart), update $F$.
\RETURN defining forms $\{F_{1},\dots,F_{m-1}\}$ of the resulting curve.
\end{algorithmic}
\end{algorithm}

\begin{proposition}
SmoothModel terminates and outputs a curve satisfying
Theorem~\ref{thm:smooth}.  Each blow-up multiplies
$\|F\|_{\infty}$ by at most $n^{4}$.
\end{proposition} 