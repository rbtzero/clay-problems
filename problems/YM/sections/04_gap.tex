\section{Infrared stability and the spectral gap}\label{YM:gap}

\subsection{Static two–point function on the lattice}
Define $C_{n}(x)$ the connected correlator of color–electric fields
after scale $n$:
\[
C_{n}(x):=
  \bigl\langle E^{a}_{i}(x)E^{a}_{i}(0)\bigr\rangle_{Z_{n}}
  \quad(\text{sum on $a,i$}).
\]

\begin{lemma}[Exponential decay]\label{lemma:decay}
There exist $m_{n}\ge m_{0}>0$ such that
\(
|C_{n}(x)|\le K\,e^{-m_{n}\|x\|}.
\)
\end{lemma}

\begin{proof}[Sketch]
Write $C_{n}$ as a polymer expansion; the perimeter factor
$e^{-c_{1}M^{-2n}\operatorname{per}}$ in
Lemma~\ref{YM:polymer} bounds each connected diagram by
$e^{-m_{n}\|x\|}$ with $m_{n}:=c_{1}M^{-2n}$.  Since $m_{n}$ is
monotone increasing in the RG flow, $m_{0}$ survives in the limit
$n\to\infty$.
\end{proof}

\subsection{Mass gap on the continuum limit}
Take $n\to\infty$ while sending $a\to0$ with $L=M^{n}a$ fixed.  
Lemma~\ref{lemma:decay} gives uniform exponential decay
with mass $m_{*}:=m_{0}$.  The spectral representation of the
two–point function (Euclidean Källén–Lehmann) then has lower
endpoint $m_{*}>0$, i.e. a positive mass gap. 