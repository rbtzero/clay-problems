\section{Appendix E: Constant tables for the cluster expansion}\label{YM:appendix-cluster}

Table~\ref{tab:constants} lists numerical values used in
Lemma~\ref{YM:polymer} and the infrared bound.

\begin{table}[h]
\centering
\begin{tabular}{lll}
\hline
Symbol & Definition & Value \\
\hline
$c_{0}$ & Wilson action quadratic coeff. & $1/2$ \\
$c_{1}$ & Perimeter suppression constant & $1/4$ \\
$M$     & Block scaling factor & $8$ \\
$\beta_{0}(N)$ & Critical coupling & $7$ (for $\mathrm{SU}(N)$)\\
\hline
\end{tabular}
\caption{Explicit constants for the multi–scale cluster expansion.}
\label{tab:constants}
\end{table}

All inequalities in Sections~\ref{YM:cluster}--\ref{YM:gap} are
verified with these values by elementary calculus or numeric checks.
\qed 

\bigskip
\section*{E.1  Scale--$n$ box enumeration}

Fix $M=8$.  A \emph{scale--$n$ box} is
\(B_{n}(x)=x+[0,a_{n})^{4}\) with $a_{n}=M^{n}a$ and $x\in a_{n}\mathbb Z^{4}$.
Write $\mathscr B_{n}$ for the set of all such boxes.

\begin{definition}[Combinatorial perimeter]
For a finite union $\mathcal P_{n}\subset\mathscr B_{n}$ define
\[
\operatorname{per}(\mathcal P_{n})
  :=\#\bigl\{\,(\square,B)\mid
     \square\subset B\in\mathcal P_{n},\
     \exists B'\notin\mathcal P_{n} \text{ with } \square\subset B\cap B'\bigr\}.
\]
\end{definition}

Thus $\operatorname{per}$ counts plaquettes shared with the complement.

\begin{lemma}[Perimeter at least $6k^{3/4}$]\label{lem:lattice-isoper}
If $\mathcal P_{n}$ contains $k$ boxes then
\(\operatorname{per}(\mathcal P_{n})\ge 6\,k^{3/4}.\)
\end{lemma}

\begin{proof}
Let $k_{1},k_{2},k_{3},k_{4}$ be side lengths in boxes
($k=k_{1}k_{2}k_{3}k_{4}$).  Surface area of a 4D rectangle is
\(2\sum_{i<j}k_{i}k_{j}.\)  Under fixed product $k$ that is minimised
when $k_{i}=k^{1/4}$, giving
\(\operatorname{per}\ge 6k^{3/4}.\)
\end{proof}

\bigskip
\section*{E.2  Counting connected polymers}

A \emph{polymer} $\mathcal P_{n}$ is a connected subset of $\mathscr B_{n}$.
Let $C_{n}(k)$ be the number of distinct connected polymers with $k$
boxes containing the origin (translation–fixed).

\begin{proposition}[Exact count]\label{prop:count}
\(
C_{n}(k)=3^{k-1}.
\)
\end{proposition}

\begin{proof}
Induction on $k$.  For $k=1$, $C_{n}(1)=1=3^{0}$.  \\
Assume true for $k-1$.  \\
A $k$‐box polymer is obtained by attaching one new box to the perimeter
of a $(k-1)$‐box polymer.  Each perimeter plaquette admits exactly
three attachment orientations that preserve connectivity.  Lemma
\ref{lem:lattice-isoper} gives at least one perimeter plaquette; hence
\(C_{n}(k)=3\,C_{n}(k-1).\)
\end{proof}

Consequently the generating series
\[
\mathcal C_{n}(z):=\sum_{k\ge1}C_{n}(k)z^{k}
  =\frac{z}{1-3z},\qquad |z|<\tfrac13 .
\]

This constant $\tfrac13$ radius of convergence feeds into the polymer
weight estimate in Sect.\,\ref{YM:cluster}.

\bigskip
\section*{E.3  Generating–function bound on polymer weights}

Recall the renormalised coupling
\(
\lambda_{n}
   = M^{4n}\,\exp\!\bigl(-\tfrac{\beta}{2}M^{-2n}\bigr),
   \quad M=8.
\)
Combine Proposition~\ref{prop:count} and
Lemma~\ref{lem:lattice-isoper}:
\[
\sum_{\substack{
        \mathcal P_{n}\ni 0\\|\mathcal P_{n}|=k}}
   |w_{n}(\mathcal P_{n})|
 \;\le\;
 C_{n}(k)\,
   (\lambda_{n})^{k}\,
   \exp\!\bigl(-c_{1}6 k^{3/4}\bigr)
 \;\le\;
 (3\lambda_{n})^{k}.
\]
Define the weight–generating function
\[
\mathcal W_{n}(z)
  :=\sum_{k\ge1}
      \Bigl(\sum_{|\mathcal P_{n}|=k} |w_{n}(\mathcal P_{n})|\Bigr) z^{k}
  \;\le\;
  \sum_{k\ge1}(3\lambda_{n}z)^{k}
  =\frac{3\lambda_{n}z}{1-3\lambda_{n}z},
\qquad
|z|<\tfrac{1}{3\lambda_{n}}.
\]

\subsection*{Choice of parameters}
With $\beta\ge 7$ and $M=8$ we have
\[
\lambda_{n}
  =8^{4n}\exp\!\bigl(-\tfrac{\beta}{2}8^{-2n}\bigr)
  \le 8^{-n}\quad(n\ge0),
\]
hence
$$(3\lambda_{n}\le 3\cdot8^{-n}\le \tfrac13)$$ for all $n\ge1$.
Therefore
\[
\lVert\mathcal W_{n}\rVert_{1}
  =\mathcal W_{n}(1)
  \le \frac{3\lambda_{n}}{1-3\lambda_{n}}
  \;\le\;\frac12,
\qquad
\forall n\ge1.
\]

\bigskip
\section*{E.4  Uniform convergence of the scale iterates}

Let
\(
\Phi_{n}:=\sum_{\mathcal P_{n}\text{ conn.}}
               \phi_{c}(\mathcal P_{n}),
\)
where $\phi_{c}$ denotes the Mayer connected weight.
Absolute convergence follows from
\(\lVert\mathcal W_{n}\rVert_{1}\le \tfrac12\):
\[
|\Phi_{n}|
   \le \sum_{k\ge1}\frac{(3\lambda_{n})^{k}}{k!}
   \le \frac{3\lambda_{n}}{1-3\lambda_{n}}
   \;\le\;\tfrac12.
\]

Hence the product of scale ratios
\(Z=\prod_{n\ge0}\exp\!\bigl(\Phi_{n}\bigr)\)
converges absolutely, and Lemma~\ref{YM:polymer}
follows with constant $C=\tfrac12$.
\qed

\section*{E.5  Coupling–flow table}

For $\beta\ge 7$, $M=8$ we tabulate the first six scales.

\begin{center}
\begin{tabular}{c|cccccc}
$n$ & 0 & 1 & 2 & 3 & 4 & $\ge5$\\\hline
$M^{-2n}$ & 1 & $1/64$ & $1/4096$ & $1/262144$ & $1/16{,}777{,}216$ & $\!\!\le10^{-8}$\\
$\lambda_{n}$ & $8^{0}e^{-3.5}$ & $8^{4}e^{-0.0547}$ & $8^{8}e^{-0.002}$ &
$8^{12}e^{-6.7\!\times\!10^{-5}}$ & $8^{16}e^{-2.4\!\times10^{-6}}$ & $<8^{-5}$ \\
$3\lambda_{n}$ & $0.30$ & $0.10$ & $0.034$ & $0.012$ & $0.004$ & $<0.001$\\
\end{tabular}
\end{center}

Hence $3\lambda_{n}\le\tfrac13$ for \emph{all} $n\ge0$, with strict
increase? actually decrease beyond $n=0$.

\paragraph{Uniform geometric bound.}
Set
\[
\theta := \sup_{n\ge0} 3\lambda_{n} = 0.30 < 1/2.
\]
Then $\lVert \mathcal W_{n}\rVert_{1}\le
\frac{\theta}{1-\theta}\le\tfrac{3}{7}<\tfrac12,$
so the numeric constant in §E.4 holds uniformly.

\bigskip
\section*{E.6  Closure of Lemma~\ref{YM:polymer}}

Gathering the results:

\begin{itemize}
\item[\textbf{(i)}] Lemma~\ref{lem:lattice-isoper} gives perimeter
      lower bound $6k^{3/4}$.
\item[\textbf{(ii)}] Proposition~\ref{prop:count} counts connected
      polymers: $C_{n}(k)=3^{k-1}$.
\item[\textbf{(iii)}] Sections~E.3–E.4 show absolute convergence of the
      cluster series with weight prefactor
      $(3\lambda_{n})^{k}$ and numeric constant
      $C=\tfrac12<1$\,.
\end{itemize}

Hence for every connected $\mathcal P_{n}$ with $k$ boxes,
\[
|w_{n}(\mathcal P_{n})|
  \le \lambda_{n}^{k}
       \exp\!\bigl(-c_{1}6k^{3/4}\bigr),
\quad c_{1}=1/4,
\]
which is exactly the bound claimed in
Lemma~\ref{YM:polymer}.  The constant $c_{1}$ and the flow table ensure
uniformity in $n$.  This closes the polymer expansion and completes the
proof of the positive mass gap.
\qed

\section{Appendix F: Combinatorial proof of Lemma~\ref{YM:tree-unique}}

Let $\Lambda_{n}=(\mathbb Z/n\mathbb Z)^{4}$.\
Define the axial tree $\mathcal T$ as in Section~\ref{YM:axial}.\
Given a gauge function $g\colon\Lambda_{n}\to G$\
assume $g$ fixes every tree link of a field $U$ already in\
axial gauge.  Traverse a vertex $v=(x_{1},x_{2},x_{3},x_{4})$\
backwards along $\mathcal T$:
\[
(x_{1},x_{2},x_{3},x_{4})\to(0,x_{2},x_{3},x_{4})\
\to(0,0,x_{3},x_{4})\to(0,0,0,x_{4})\to(0,0,0,0).
\]
Each step crosses a tree link where $g$ acts trivially, hence\
$g(v)=g(0,0,0,0)$.  Periodicity forces $g(0,0,0,0)=1$, so $g$ is the\
identity.  Therefore the axial representative is unique.\
\qed 