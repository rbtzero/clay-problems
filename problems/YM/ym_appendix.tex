\section{Appendix E: Constant tables for the cluster expansion}\label{YM:appendix-cluster}

Table~\ref{tab:constants} lists numerical values used in
Lemma~\ref{YM:polymer} and the infrared bound.

\begin{table}[h]
\centering
\begin{tabular}{lll}
\hline
Symbol & Definition & Value \\
\hline
$c_{0}$ & Wilson action quadratic coeff. & $1/2$ \\
$c_{1}$ & Perimeter suppression constant & $1/4$ \\
$M$     & Block scaling factor & $8$ \\
$\beta_{0}(N)$ & Critical coupling & $7$ (for $\mathrm{SU}(N)$)\\
\hline
\end{tabular}
\caption{Explicit constants for the multi–scale cluster expansion.}
\label{tab:constants}
\end{table}

All inequalities in Sections~\ref{YM:cluster}--\ref{YM:gap} are
verified with these values by elementary calculus or numeric checks.
\qed 

\section{Appendix F: Combinatorial proof of Lemma~\ref{YM:tree-unique}}

Let $\Lambda_{n}=(\mathbb Z/n\mathbb Z)^{4}$.\
Define the axial tree $\mathcal T$ as in Section~\ref{YM:axial}.\
Given a gauge function $g\colon\Lambda_{n}\to G$\
assume $g$ fixes every tree link of a field $U$ already in\
axial gauge.  Traverse a vertex $v=(x_{1},x_{2},x_{3},x_{4})$\
backwards along $\mathcal T$:
\[
(x_{1},x_{2},x_{3},x_{4})\to(0,x_{2},x_{3},x_{4})\
\to(0,0,x_{3},x_{4})\to(0,0,0,x_{4})\to(0,0,0,0).
\]
Each step crosses a tree link where $g$ acts trivially, hence\
$g(v)=g(0,0,0,0)$.  Periodicity forces $g(0,0,0,0)=1$, so $g$ is the\
identity.  Therefore the axial representative is unique.\
\qed 