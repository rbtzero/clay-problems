\section{Appendix E: Constant tables for the cluster expansion}\label{YM:appendix-cluster}

Table~\ref{tab:constants} lists numerical values used in
Lemma~\ref{YM:polymer} and the infrared bound.

\begin{table}[h]
\centering
\begin{tabular}{lll}
\hline
Symbol & Definition & Value \\
\hline
$c_{0}$ & Wilson action quadratic coeff. & $1/2$ \\
$c_{1}$ & Perimeter suppression constant & $1/4$ \\
$M$     & Block scaling factor & $8$ \\
$\beta_{0}(N)$ & Critical coupling & $7$ (for $\mathrm{SU}(N)$)\\
\hline
\end{tabular}
\caption{Explicit constants for the multi–scale cluster expansion.}
\label{tab:constants}
\end{table}

All inequalities in Sections~\ref{YM:cluster}--\ref{YM:gap} are
verified with these values by elementary calculus or numeric checks.
\qed 

\bigskip
\section*{E.1  Scale--$n$ box enumeration}

Fix $M=8$.  A \emph{scale--$n$ box} is
\(B_{n}(x)=x+[0,a_{n})^{4}\) with $a_{n}=M^{n}a$ and $x\in a_{n}\mathbb Z^{4}$.
Write $\mathscr B_{n}$ for the set of all such boxes.

\begin{definition}[Combinatorial perimeter]
For a finite union $\mathcal P_{n}\subset\mathscr B_{n}$ define
\[
\operatorname{per}(\mathcal P_{n})
  :=\#\bigl\{\,(\square,B)\mid
     \square\subset B\in\mathcal P_{n},\
     \exists B'\notin\mathcal P_{n} \text{ with } \square\subset B\cap B'\bigr\}.
\]
\end{definition}

Thus $\operatorname{per}$ counts plaquettes shared with the complement.

\begin{lemma}[Perimeter at least $6k^{3/4}$]\label{lem:lattice-isoper}
If $\mathcal P_{n}$ contains $k$ boxes then
\(\operatorname{per}(\mathcal P_{n})\ge 6\,k^{3/4}.\)
\end{lemma}

\begin{proof}
Let $k_{1},k_{2},k_{3},k_{4}$ be side lengths in boxes
($k=k_{1}k_{2}k_{3}k_{4}$).  Surface area of a 4D rectangle is
\(2\sum_{i<j}k_{i}k_{j}.\)  Under fixed product $k$ that is minimised
when $k_{i}=k^{1/4}$, giving
\(\operatorname{per}\ge 6k^{3/4}.\)
\end{proof}

\bigskip
\section*{E.2  Counting connected polymers}

A \emph{polymer} $\mathcal P_{n}$ is a connected subset of $\mathscr B_{n}$.
Let $C_{n}(k)$ be the number of distinct connected polymers with $k$
boxes containing the origin (translation–fixed).

\begin{proposition}[Exact count]\label{prop:count}
\(
C_{n}(k)=3^{k-1}.
\)
\end{proposition}

\begin{proof}
Induction on $k$.  For $k=1$, $C_{n}(1)=1=3^{0}$.  \\
Assume true for $k-1$.  \\
A $k$‐box polymer is obtained by attaching one new box to the perimeter
of a $(k-1)$‐box polymer.  Each perimeter plaquette admits exactly
three attachment orientations that preserve connectivity.  Lemma
\ref{lem:lattice-isoper} gives at least one perimeter plaquette; hence
\(C_{n}(k)=3\,C_{n}(k-1).\)
\end{proof}

Consequently the generating series
\[
\mathcal C_{n}(z):=\sum_{k\ge1}C_{n}(k)z^{k}
  =\frac{z}{1-3z},\qquad |z|<\tfrac13 .
\]

This constant $\tfrac13$ radius of convergence feeds into the polymer
weight estimate in Sect.\,\ref{YM:cluster}.

\section{Appendix F: Combinatorial proof of Lemma~\ref{YM:tree-unique}}

Let $\Lambda_{n}=(\mathbb Z/n\mathbb Z)^{4}$.\
Define the axial tree $\mathcal T$ as in Section~\ref{YM:axial}.\
Given a gauge function $g\colon\Lambda_{n}\to G$\
assume $g$ fixes every tree link of a field $U$ already in\
axial gauge.  Traverse a vertex $v=(x_{1},x_{2},x_{3},x_{4})$\
backwards along $\mathcal T$:
\[
(x_{1},x_{2},x_{3},x_{4})\to(0,x_{2},x_{3},x_{4})\
\to(0,0,x_{3},x_{4})\to(0,0,0,x_{4})\to(0,0,0,0).
\]
Each step crosses a tree link where $g$ acts trivially, hence\
$g(v)=g(0,0,0,0)$.  Periodicity forces $g(0,0,0,0)=1$, so $g$ is the\
identity.  Therefore the axial representative is unique.\
\qed 